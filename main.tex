%%
% The BIThesis Template for Bachelor Paper Translation
%
% 北京理工大学毕业设计(论文) —— 使用 XeLaTeX 编译
%
% Copyright 2020-2023 BITNP
%
% This work may be distributed and/or modified under the
% conditions of the LaTeX Project Public License, either version 1.3
% of this license or (at your option) any later version.
% The latest version of this license is in
%   http://www.latex-project.org/lppl.txt
% and version 1.3 or later is part of all distributions of LaTeX
% version 2005/12/01 or later.
%
% This work has the LPPL maintenance status `maintained'.
%
% The Current Maintainer of this work is Feng Kaiyu.
%
% Compile with: xelatex -> biber -> xelatex -> xelatex
%%

% !TeX program = xelatex
% !BIB program = biber


\documentclass[type=bachelor_translation]{bithesis}

% 此处仅列出常用的配置。全部配置用法请见「bithesis.pdf」手册。
\BITSetup{
  cover = {
    % 在封面中载入有「北京理工大学」字样的图片,如无必要请勿改动。
    headerImage = images/header.png,
    % 在封面标题中使用思源黑体,使用此选项可以保证与 Word 封面标题的字体一致。
    xiheiFont = STXIHEI.TTF,
    % 官方模板采用了固定的下划线宽度。我们采用以下两个选项来达成这个效果。
    % 如果你想要使用自动计算的下划线宽度,也可以删去以下两个选项。
    autoWidth = false,
    valueMaxWidth = 20em,
  },
  info = {
    title = 跨操作系统的异步音频驱动模块设计与实现,
    titleEn = {Design and Implementation of Cross-Platform Asynchronous Audio and Video Driver Module},
    % 注意,这里要写的是毕设的大标题,不是你要翻译的文献的标题。
    % 想要删除某项封面信息,直接删除该项即可。
    % 想要让某项封面信息留空(但是保留下划线),请传入空白符组成的字符串,如"{~}"。
    % 如需要换行,则用 “\\” 符号分割。
    school = 计算机学院,
    major = 计算机科学与技术,
    author = 穆新宇,
    class = 07112005,
    studentId = 1120202695,
    supervisor = 陆慧梅,
    translationTitle = 使用Rust语言编写网卡驱动,
    translationOriginTitle = Writing Network Drivers in Rust,
    keywords = {北京理工大学;本科生;毕业设计(外文翻译);Rust;网卡驱动;},
    % 如果你的毕设为校外毕设,请将下面这一行语句解除注释(删除第一个百分号字符)并填写你的校外毕设导师名字
    % externalSupervisor = 左偏树,
  },
  style = {
    % head = {自定义页眉文字}
    % 开启该选项后,将用 Times New Roman 的开源字体 TeX Gyre Termes 作为正文字体。 
    % 这个选项适用于以下情况:
    % 1. 不想在系统中安装 Times New Roman。
    % 2. 在 Linux/macOS 下遇到 `\textsc` 无法正常显示的问题。
    % betterTimesNewRoman = true,
  }
}

% 大部分关于参考文献样式的修改,都可以通过此处的选项进行配置。
% 详情请搜索「biblatex-gb7714-2015 文档」进行阅读。
\usepackage[
  backend=biber,
  style=gb7714-2015,
  gbalign=gb7714-2015,
  gbnamefmt=lowercase,
  gbpub=false,
  doi=false,
  url=false,
  eprint=false,
  isbn=false,
]{biblatex}

\usepackage{minted}% 语法高亮和代码样式设置方面更加强大和灵活
\usepackage{listings}% 引入listings包,用于在文档中插入代码,并可自定义代码样式

% 参考文献引用文件位于 misc/ref.bib
\addbibresource{misc/ref.bib}

% 文档开始
\begin{document}

% 标题页面:如无特殊需要,本部分无需改动
% \input{misc/0_cover.tex}
\MakeCover

% 前置页面定义
\frontmatter
%\input{misc/1_originality.tex}
% 摘要:在摘要相应的 TeX 文件处进行摘要部分的撰写
%%
% The BIThesis Template for Bachelor Paper Translation
%
% 北京理工大学毕业设计(论文) —— 使用 XeLaTeX 编译
%
% Copyright 2020-2023 BITNP
%
% This work may be distributed and/or modified under the
% conditions of the LaTeX Project Public License, either version 1.3
% of this license or (at your option) any later version.
% The latest version of this license is in
%   http://www.latex-project.org/lppl.txt
% and version 1.3 or later is part of all distributions of LaTeX
% version 2005/12/01 or later.
%
% This work has the LPPL maintenance status `maintained'.
%
% The Current Maintainer of this work is Feng Kaiyu.
%
% Compile with: xelatex -> biber -> xelatex -> xelatex
%%

% 中文摘要
\begin{abstract}
% 中文摘要正文从这里开始
许多开发者认为编写网络驱动程序是一项不愉快的任务。主要有三个原因导致他们不愿意这么做:在内核空间工作的困难性,大多数驱动程序的复杂性以及对C语言开发的普遍不情愿。

然而,现在有其他选择。内核模块不必一定用C语言编写,现代网卡支持越来越多的硬件委托功能,可以大大简化驱动程序的复杂度。用户空间网络驱动程序的兴起也避免了编写内核代码的需要。

我们展示了一个用Rust编写的先进用户空间网络驱动程序,该驱动程序注重简单性、安全性和性能,从而证明了驱动程序开发既具有挑战性又具有回报性。这个驱动程序总行数1306行,不安全代码占比不到10\%,它注重核心的报文处理(packet processing)功能,但在单个3.3GHz的CPU核心上,转发(forwarding)性能高达每秒超过2600万包,超过内核及其他一些用户空间驱动程序。

我们讨论了我们的实现细节,从不同角度对这个驱动进行评估。从结果来看,我们得出结论——Rust是否适合编写网络驱动程序。

\end{abstract}

% 目录
\MakeTOC

% 正文开始
\mainmatter

% 第一章
% 第一章

\chapter{介绍}

近年来,越来越多的开发者将驱动程序从内核空间移植到用户空间。与网络驱动相关的这个趋势背后的一个主要原因,就是套接字API的性能瓶颈问题。通用内核栈对现代需求来说简直太慢了。过去,开发者为规避这个问题而自己编写内核驱动程序。但是,内核驱动开发是一个繁琐的过程,因为在这么低级的代码层面上出现任何编程错误都可能和将导致内核崩溃。此外,内核还对开发环境和内核中可用工具施加各种限制。

幸运的是,内核驱动开发有其他选择。在第二章中,我们通过展示Linux中低级网络通信的不同方法来说明这些选择。我们还谈到用户空间驱动的优点之一就是可以选择任何编程语言进行实现。但是,大多数用户空间驱动仍然用C语言编写,这是一种早已过时的语言,如果不小心处理的话很容易导致缓冲区溢出、堆栈溢出、段错误、内存泄漏等未定义行为的问题。由于可以选择任何编程语言,那么就产生了一个问题——哪种编程语言特别适合网络驱动开发?

为回答这个问题,我们在第三章中讨论了网络驱动编程语言应该具备的优良属性,并建议选择Rust这一先进编程语言来替代C进行网络编程。Rust能够满足我们所有理想的特性。我们通过介绍Rust的核心特性和一些独特的Rust概念来详细讨论Rust。
% 第二章
% 第二章

\chapter{Linux上的网络通信}

虽然互联网如今无处不在,但网络通信、网卡和网络驱动程序仍然经常被开发人员和用户视为黑箱。对于用户来说,大多数时候,他们只需要将网线插入电脑就可以上网了。对于开发人员来说,操作系统提供的高级API让他们可以轻松地在其程序中包含网络通信功能,而无需处理低级网络编程的复杂性。

然而,如果还没有为某种网卡提供驱动程序,或者当报文处理性能成为一个重要考虑因素时,有必要对网络通信的基础知识,尤其是应用程序与操作系统和网卡交互的方式,有初步的了解。每次从应用程序发送数据时,数据都会经过几层处理,直到到达网卡。

\begin{figure}[htbp]
    \centering
    \includegraphics[width=0.5\textwidth]{images/figure2-1.eps}
    \caption{OSI参考模型中的通信层次}\label{fig:OSI-layers} % label 用来在文中索引
\end{figure}

一个好的模型——也是事实上的标准——是OSI参考模型,OSI模型描述了这些层次(layers),它的结构如图\ref{fig:OSI-layers}所示。

模型中的每一层都为上层提供不同的服务,如数据包分割(第4层)、路由(第3层)或可靠传输(第2层)。

通常,应用程序负责应用层、表示层和会话层,而传输层和网络层则由操作系统通过通用内核网络堆栈进行处理。最低的两层,物理层和数据链路层,由网络接口卡(NIC)和网络驱动程序控制。

由于Linux操作系统的设计,传统上,网络应用程序和网络驱动程序的不同任务在Linux中被用户空间和内核空间分隔开。然而,Linux中有不同的低级数据包处理方法:大多数应用程序使用内核的套接字API,而一些应用程序运行自己的内核模块,一些在内核空间中处理所有内容,一些在用户空间中处理所有内容。

\section{内核空间}

Linux是基于宏内核的操作系统,这意味着整个操作系统都在内核空间中运行,而其他所有服务都在用户空间中运行。内核为用户空间应用程序提供套接字API,即各种函数和数据类型,用于在所谓的\textit{Berkeley}或\textit{POSIX}套接字上提供易于使用的接口。这些套接字是特殊文件,可以通过简单的读写操作与本地进程或远程主机进行通信,遵循Unix的“一切都是文件”的概念。希望建立通信通道的用户空间应用程序会调用\verb|socket()|函数,该函数最终会在内核中创建一个套接字结构实例,并向应用程序返回该套接字的文件描述符。随后,\verb|bind()|和\verb|connect()|或\verb|listen()|和\verb|accept()|将两个或多个进程之间的连接与套接字关联起来,\verb|send()|和\verb|recv()|则用于向套接字发送和接收数据。

为了减小通用内核堆栈的问题,一些应用实现了它们自己的内核模块。对于这种方式,两个著名的例子是Open vSwitch\cite{pfaff2015design}和Click Modular Router\cite{kohler2000click}。


% \input{chapters/3_chapter3.tex}

% 后置部分
\backmatter

% 结论:在结论相应的 TeX 文件处进行结论部分的撰写
%%
% The BIThesis Template for Bachelor Paper Translation
%
% 北京理工大学毕业设计(论文) —— 使用 XeLaTeX 编译
%
% Copyright 2020-2023 BITNP
%
% This work may be distributed and/or modified under the
% conditions of the LaTeX Project Public License, either version 1.3
% of this license or (at your option) any later version.
% The latest version of this license is in
%   http://www.latex-project.org/lppl.txt
% and version 1.3 or later is part of all distributions of LaTeX
% version 2005/12/01 or later.
%
% This work has the LPPL maintenance status `maintained'.
%
% The Current Maintainer of this work is Feng Kaiyu.
%
% Compile with: xelatex -> biber -> xelatex -> xelatex
%%

\begin{conclusion}
  % 结论部分尽量不使用 \subsection 二级标题,只使用 \section 一级标题

  % 这里插入一个参考文献,仅作参考
  本文结论……。\cite{李成智2004飞行之梦}

  \textcolor{blue}{结论作为毕业设计(论文)正文的最后部分单独排写,但不加章号。结论是对整个论文主要结果的总结。在结论中应明确指出本研究的创新点,对其应用前景和社会、经济价值等加以预测和评价,并指出今后进一步在本研究方向进行研究工作的展望与设想。结论部分的撰写应简明扼要,突出创新性。阅后删除此段。}

  \textcolor{blue}{结论正文样式与文章正文相同:宋体、小四;行距:22 磅;间距段前段后均为 0 行。阅后删除此段。}
\end{conclusion}

% 参考文献:如无特殊需要,参考文献相应的 TeX 文件无需改动,添加参考文献请使用 BibTeX 的格式
%   添加至 misc/ref.bib 中,并在正文的相应位置使用 \cite{xxx} 的格式引用参考文献
%%
% The BIThesis Template for Bachelor Paper Translation
%
% 北京理工大学毕业设计(论文) —— 使用 XeLaTeX 编译
%
% Copyright 2020-2023 BITNP
%
% This work may be distributed and/or modified under the
% conditions of the LaTeX Project Public License, either version 1.3
% of this license or (at your option) any later version.
% The latest version of this license is in
%   http://www.latex-project.org/lppl.txt
% and version 1.3 or later is part of all distributions of LaTeX
% version 2005/12/01 or later.
%
% This work has the LPPL maintenance status `maintained'.
%
% The Current Maintainer of this work is Feng Kaiyu.
%
% Compile with: xelatex -> biber -> xelatex -> xelatex
%%

\begin{bibprint}
  \printbibliography[heading=none]
\end{bibprint}

% 附录:在附录相应的 TeX 文件处进行附录部分的撰写
%%
% The BIThesis Template for Bachelor Paper Translation
%
% 北京理工大学毕业设计(论文) —— 使用 XeLaTeX 编译
%
% Copyright 2020-2023 BITNP
%
% This work may be distributed and/or modified under the
% conditions of the LaTeX Project Public License, either version 1.3
% of this license or (at your option) any later version.
% The latest version of this license is in
%   http://www.latex-project.org/lppl.txt
% and version 1.3 or later is part of all distributions of LaTeX
% version 2005/12/01 or later.
%
% This work has the LPPL maintenance status `maintained'.
%
% The Current Maintainer of this work is Feng Kaiyu.
%
% Compile with: xelatex -> biber -> xelatex -> xelatex
%%

\begin{appendices}
  附录相关内容…

  % 这里示范一下添加多个附录的方法:

  \section{\LaTeX 环境的安装}
  \LaTeX 环境的安装。

  \section{BIThesis 使用说明}
  BIThesis 使用说明。

  \textcolor{blue}{附录是毕业设计(论文)主体的补充项目,为了体现整篇文章的完整性,写入正文又可能有损于论文的条理性、逻辑性和精炼性,这些材料可以写入附录段,但对于每一篇文章并不是必须的。附录依次用大写正体英文字母 A、B、C……编序号,如附录 A、附录 B。阅后删除此段。}

  \textcolor{blue}{附录正文样式与文章正文相同:宋体、小四;行距:22 磅;间距段前段后均为 0 行。阅后删除此段。}

\end{appendices}



\end{document}
